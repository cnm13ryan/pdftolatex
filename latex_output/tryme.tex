
\documentclass[a4paper,12pt]{article}

\usepackage{amsmath}

\usepackage{amssymb}

\usepackage{graphicx}

\usepackage[margin=1in]{geometry}

\setlength{\parindent}{0pt}

\graphicspath{/Users/nokman/git/pdftolatex/src/../localstore/trymeassets}

\begin{document}

    Ex 3.1.1 (Mctfenpt )  Let a,b¢,d be objects such thet. 5a,b\% = Fe, 2.    Show thet. af Jeost one ft The tio statements "Q=€ and fed .  and. “acd and bec ” hold .    /. Suppose none of The statements held, st. yen 3a,b 3 = 3,4).  9 [ laze) (b-d)JVE Ged) Albee) |  [G\#9 vibed) (AL Gee) v (bec) |  2. From step l, ae 3a,b3 , ag Ec,d3, Smee a\#C and akd.  be Fabs » bf §c,d3\_ since BFC aod SFA.  Similedy fe Cc and d. oe CorFadletion .    At fest one sfalemert shoald hold 

\vspace{10pt}

\par

\vspace{10pt}

\begin{figure}[h]

\includegraphics[width=\textwidth]{/Users/nokman/git/pdftolatex/src/../localstore/trymeassets/0.jpg}

\centering

\end{figure}

\par

\vspace{10pt}

    Ex 3.1.3 (]"atlengt )   Frove Lemma 3.1.12   ya a and b are object. Then Sab = ja3U0 \$b3.   7 A,B, and C are sels. Then union operation is commutative. and associative .  Abo, we have AVA = Avg = DUA =A .    ae £a,b3, ae €a3U \$b3 Vire Vesa. f ae F233, ae §23V0 \$b3,    /.  be fabs, be \$03U \%b3, LE €a-b3,  if be ¢b3, be 4a3VGb3.  be Fa,b3.  2. Sow that AUB = BUA  tt ee AUB. Then xe A eo xe B.  Tt xe A. Th, x© BUA  3. Show thet (AuB)uc = AU (Buc)  @)tf x € (AUB)UC Than xe AUB ow xe C  Tf x€C, Thn x € BUC.  Zt xe AUB. Tha xeAow x eb.  TH « cA. Tle wc A. Ths  LHS +RH.S.    af xe B. The x @ BUC.   Tf xE AU (vc). Tr xe Aor xe BUC  TExe A. Than xe AUB   Hx é€ AvB, Tle xe A of x € 8.  TfxeA, xa AvB   ThxeB, xe AB.   Tye BUC. Tm xe B wo XE C   TH x zB. Jew xe AVP.   Zt xe Cc. Thr KEC.    (~\&    we    4 Tf xe AUD. Th veh Ts AvA = Aud: GA =A. 

\vspace{10pt}

\par

\vspace{10pt}

    Ex 3,14 Clattenpt)   Prove Js poston 3/. /7, (Sets are. portially ordered by set inclusion) .  Let A,B,C be sets. Ff ASB amd Bec. Tn ASC.   Tf AcR ad BSA: The A=B.   af AcB and Bel. Ten Ac CC.    | xehk, ASB = xeB,  Bel, 2 xe Cl, ts. xeAa xec, ASC,    2. xeA ASB, xe B.  xEB SA, ~eA, Ths, A=B. (Axiom 3.2)    3. xe AcB, xe B  BeA. xe, Acc. AAC. 

\vspace{10pt}

\par

\vspace{10pt}

\begin{figure}[h]

\includegraphics[width=\textwidth]{/Users/nokman/git/pdftolatex/src/../localstore/trymeassets/1.jpg}

\centering

\end{figure}

\par

\vspace{10pt}

\begin{figure}[h]

\includegraphics[width=\textwidth]{/Users/nokman/git/pdftolatex/src/../localstore/trymeassets/2.jpg}

\centering

\end{figure}

\par

\vspace{10pt}

\begin{figure}[h]

\includegraphics[width=\textwidth]{/Users/nokman/git/pdftolatex/src/../localstore/trymeassets/3.jpg}

\centering

\end{figure}

\par

\vspace{10pt}

    Ex 3.).] (attempt)  Let A,B,C be sets. So that ANB SA, ANB CB,  Ficthermore, Show that C£A and CEB H Cc SZ ANB.    Tho similar spirit, show That Ac AUR, Be AUB.  Ficthermore , show that. A € Cand Be Cc a AVB EC.    [ xe ANB = ~€A od xe B. Ths ANB © A md ANB EB.    2. @) vf cfA and CER The xe C, xeAand xeB Ths CEAR,  (<7) Tt CCAAB. Then xe C. xe ANB > xe A and xe BB. from  above ANBSA ad AnBe B.    3. Zf xe A. Then xe A oe xe B. Ths xe AUB. ASAUB.  Gf xe B. Then xe Ace xe 8B Ths xe AUB Be AUB.    4. (>) DEAS Cad BEC. web xe l    => AUBS C.  AE B 7xé C.    le) Tf AvB s C. Thn xe AVB xe C    Ii xcA, «eC 4. AEC, BLC.  af xe B, KE C. 

\vspace{10pt}

\par

\vspace{10pt}

\begin{figure}[h]

\includegraphics[width=\textwidth]{/Users/nokman/git/pdftolatex/src/../localstore/trymeassets/4.jpg}

\centering

\end{figure}

\par

\vspace{10pt}

\begin{figure}[h]

\includegraphics[width=\textwidth]{/Users/nokman/git/pdftolatex/src/../localstore/trymeassets/5.jpg}

\centering

\end{figure}

\par

\vspace{10pt}

    Ex 3.Llo C(athimgt )  Let Aand B be sets,    Show That the Three sel A \B, AAB and B\A are dis om,  Sho» thet Tele union is AUB.    LAKEA\B = «eA md xf BD xe A\ (ANB) .,  => X¢ B\A x ¢é (ANB)    Tes @\8)n AaB) = 9  (A) A B\A = ©.    2 Tf xe ANB. Then ve A ad xeB = xg B\A.  (ANB) a (B\A) = @.    c=)  3. (A\B) VU (B\A) > xe A and x¢B    on =) x¢ Gus) \Aas).  xeBSad xgA.  (tare) (8xA) U had) > xe (Av9)\ An) LU An)  x E AVE.    \&) x € AVB, =) “<¢ Ao £61 ee XE AN B.  x EC ANR of xe B\A of xe AA, 

\vspace{10pt}

\par

\vspace{10pt}

    nd st  Ex 2.1-U (2 a thept, | offen ¢é abet\}  Show that The axjo ot re Jecement implies te AKiom f specifeaton.  CA«' )    ixtem3- inm 3.6  Axiom 3.71 Replacemect)  Let A ben set for any Het x eA, and. any object Yr Sppose\_we have « stifemen\{  Plx,y) pring 4 x andl y st for enh x eA, Bere is of most one Y fre hick  Pl x,y) is Tree. Then there existe a sel Fy: Ply) is re por some xeA\$,  8b, Aor any object z, ZeE dy: Plx,y) is True Soe Some Xe AS  >) Plx2) is Tee fer some xe A.  Axiom 3-6 Gprerfiatin)  Let A be o set, and Fir each xe A, ket Pi) leu poopy prbing Ta x  Cie, Ae cack xe A, Ple) is etther Five or tbe). Then there exits a set.  SxeA: P\&S) is treed whose element ave recisely the okmeits x in A  soe which Pl) is Troe. Za otter works, or any object y,  ye gxeA: P\&) is Tre 3 <=> lyeA andl PG) ws Tue).    ZE By: Px y) is Tme foc sume x € A3    Suppose we modity the ahve s-T DE Z yi Puy) +s Five f- any x€ At.  suf.    3 Y- Py) ss Tue fir some XC A3 c gy: Plx,y) is re fir ony xe A’  a  Ey:y e 3xeA: Pe)\$ y  2. ths, fam step \, f 2E Sy: Ply) Laxeh3,  Then 26 Sy: ye §xe A: PGBS.  s.-T. Replacemestt Implees SpoctheLion. 

\vspace{10pt}

\par

\vspace{10pt}

\begin{figure}[h]

\includegraphics[width=\textwidth]{/Users/nokman/git/pdftolatex/src/../localstore/trymeassets/6.jpg}

\centering

\end{figure}

\par

\vspace{10pt}

    Ex 31. 1Qii Cl atfengt )  Sup pose A, 8, A’, 3° are sets Lst. A'sA and B’< B.    (ii) Give a Coviler — excample shoumg A’\B < A\8 is falta.  Gun yeu fad a modifies fon of th slefement imelving the set d:Hlerence gperoctem \  which is Tine ye the shied bypothess?,    [| Zf xe A'\B. Then xeA and x¥ B  Jf xe AN. Than xe A  The g Bl. Then xe Ro xf B, Sire BEB.  If x€B' ond Kx¢B. Ten xe A\B  Tf x ¢ Bl and xc i3. The xe A\B’.    2. A \B’ cS A\ BL “The jest ticthn is aS oh~e | 

\vspace{10pt}

\par

\vspace{10pt}

    E3113 Clettengt)  DeFine a Pups wbset fa seb A b fe a sbset Bf Avi BEA.  Le€ A be a non-empty sel  Show That A does aot have any pon- empty Pepe shee he  As The Poem A: 5x3 x Some beet x,    |, Sach et Cis C= FxeB: BEAS    2. (=?) aK eA, st. KE C , (A does rot hove any hon - empty proper subsets).    Aye C, y FB,  dee A, st xe C, x¢ D Thess. Az €\%3 Xr some abject x,    (<=) zt Ais d te fon Az \{x3 fe some odject x,  Tien FxeA, x¢ DB (Ais non-empty \$0) »  x\#D,    Be- C . Ton ay € x= C, 3st CSA. 

\vspace{10pt}

\par

\vspace{10pt}

    Ex 3.2.| ((“athmgt]  Shows tot He. anivecse| spe ifmtiom axiom Ailes 3.9, if assumed t le true,  would imphy Axioms 2.3, 3.4,3.5, 3.6, ard 3.) ( Asjom 3\% if we assume alfn fom    number\$ ave ects ).  Axiom 3.3. (Eupty set)    Axiom 3-9 (Usinecsel Sperfictvs) on 3 Geyttn edwin)  Sipe for ery abject x, we hae « popely PG) patinig fe x Ain 35S)  Then Tere exis a set Ex: Pl)’ sé we every abject a Adon 37 (Reylwznent\   y€ §x: PG)’ <=> PY) is tree. Axion 3.8 (Tetoit\    [ xe § y: 76935 <9 xe D Ch os empty). =) x is al an object.    2. xe 3 y: yeas <=> X=a C Sime set)  KE gy: Cy= a) v (y=8)3 a) (= a) oF («+4) Pair st)    3. x€ AUB es xe Sy: VeA)v (yeB)5 <=> eA) or eB) (Ric Vinn),    xeSy: YeAAPG)3 <= eA A Pe) pens tati. )    as    Na    2¢€ by: Py) fe xe A 3. <\_ PX, 2) fr Some xeA. Repluerct)    LZ. xe sntt: neIN3 ced x é WN. ( Tort, J, 

\vspace{10pt}

\par

\vspace{10pt}

    Ex 3.2.2 (("attenpt|    Use the axiom of cephoty Cand the singleton set axtom) tb thw thet f  Ais aset, Hen A @A. Fectheroace show thet tA ad Bae fuo se,  Tien er Ther A€B8 or BEA Ge oh).    Then pee the. corollary ; briven any get A, there exist a malhemefical object  Fiat is not an emot jn A, nanely A ite Ths. \# biger sel com de erected by AV GAS.    | BA a sot, Then by Arm 3-10 (Kepke), Ax \$x: xed\}  AxecA, x= Y where ye gx: xe A\}. Axiom 3.4 (Salt sc\}    or    xNA= Z.  “Thos A: 3x: xe A\$ ¢ A.    2. Bt A ad B ore feo sels. Thn Az fx: xe A\}    B: Sy: ye B3.  Suppose A\&B.    Then deze iv: y 6 83, Zz: gx: xeA3,  Z2NB= 2. yb Greil Axiom 3.10 (Regslenty). anB-PVadgB  Ths A€B. (this also applies BEA)    3. Ay sf, A’=A and A’ €A, Then it ders mt cotediel Axism3.(0 (Regulasty)  AU SA'S is possible .  But not AUA', sine AUA’=A= A’ 

\vspace{10pt}

\par

\vspace{10pt}

    Fx3.22 (I “atlenpt)  Show thet the universal speerFieaTion axiom , Axiom 3. 7, 6 equivalent b  an axiom post ling He existence tf a “wniversel set” SL consisting Af all objects    Tp other wos, chev thet Axion 3.9 is Tue itt a unnersn/ sel exish.    Farthermore, show that b Axlin 3.1, SU e Q. cotbnd ith Be?.2.2.  Then pustily hy Axle 3-7 Js exe[vded fim Wim T recabelly.    | (<) SL = \$x: Pe), = § xs xisaset and x¢x§ (The unnersal set)    Suppose if SL exist. Then \# 44] By Trae , Het is X ts « set and x fx.  Then x GS,    HY xen.  Jen xX is a sel and X é x, setubty 7),  Thus implies Axiom 3.9.    (=) Swprse Axrem 3.4 1s Tre. Then if Fly) ws tbe y tS oa st end yer  Ten exist , and vie verse.  Thes implies He exis/emnce - S2.    2. Since if SL is a sel Gen SL is alco an object, tes (22 €52) v(Q¢éQ).  This earhodicls the Axiom 3.10C Reg lacity) as f2€ RQ is possible .    Te have Axiom 310 (Reo lon ty ) porkirg AKrom BY CUnisercal specific.)  4 ortiely omitted . 

\vspace{10pt}

\par

\vspace{10pt}

\begin{figure}[h]

\includegraphics[width=\textwidth]{/Users/nokman/git/pdftolatex/src/../localstore/trymeassets/7.jpg}

\centering

\end{figure}

\par

\vspace{10pt}

    Ex 3.3.2 (|atlempt)  Let P:X 2 Y and 3 DE be fraslons,  Show thet it Ff ond g are both iyective . Then 50 is per    Similary Show thot it Ford 9 are doth sugjective. . Men 50 5 J of.    L Suppose Food g are inyectme tt Wx, x2 ¢ X, Vy Ye ev    Xx. «> An) \# Ae) x =x, —> \#l4) =SAln)  YAW <> 9G) \# Oh). Yaw <= gn) = 7 lv)    2. From skp I, Six yi = 4.) 2 Yrn> Slr)  and we knew that. Yi = Ye => IM) = Gf)G:) = Gh) = (g°F 0)  y \# \%) <=> Fly) = (oF) \&) \# ol) = Gof )\&)  = zt “s injective.    3. Suppose SF and J are surjective , Vy é Y, | x € XxX, s-T, He) = ¥  Vee2z, dye ¥, st, gz    > VeeZ. dye ¥, dxeX, «t 2s gly) = Gofl\&) 

\vspace{10pt}

\par

\vspace{10pt}

\begin{figure}[h]

\includegraphics[width=\textwidth]{/Users/nokman/git/pdftolatex/src/../localstore/trymeassets/8.jpg}

\centering

\end{figure}

\par

\vspace{10pt}

    Ex 3.3.4 (Matlongt |   Let A:X > ¥. fiX YX, F Y> Zz, and G2 Yo ZS be Prctone.  Show thet if get = got and 4 is inject hve | Then f:f.   Ts fhe some theme True ff JS is wel crjecThe 2   Show thet if got = Gof ard f is Suvejective Then org:   Zs the same stifemext toe \#£ Lic not Susective ®    \}. get = 5oF ingles. Vee x, Gof) \&) Gpf)O),  Ft 5 is njeethe, 60. yt PO Gln) \# gl).    \_ yep > Fl)2 gl). ae  Suppose ff, st. qxex, Ae) 4 FZ), = G-f)() \# (gf) \&)  f-L[ . Qifradeton.    2. Tle above stvbement ‘s fle if 4 Is not injecthe also ger F pet, and conn an  pove wheter fs f.    3. get * 5 ot implies Vee X, (5°) \&) 2(Fof) (x).  74 Ff is segective. “Thea Wy é y, Jxe x, 3G y? 76).  Since As Pf (Dbotty). Fer Gf)G) + Gef)O), 7\%. FY. Bz.    Wye, Bee, st Gy) = Got)\&) sme gf = 5-8,  54) = Gefl\&). Then gF9 -    4 The stilemest. is The cegodless wheter or nf fF ‘s spreitbed . 

\vspace{10pt}

\par

\vspace{10pt}

    Ex 3.3.5 (I*atlempt)  Let S:X>Y aod p> Z be tactons.  Show that if gof is injective Men S omc de injective .    TJs it Tue Thect. 4 mut abo de inective 2  Sow thet if get is Surjective. Then g mst be svejective .  Ts it Toe thet SF mst abe be. svjective *    | tf yee @-£)\&,) \# (et (Can  Suppsxe fis ast injective . Then X\#X, SS = ti) = = Sh).  = Gef166) = gph  - Cif) (6) AGN)  > Xx mst te injective . v"  g mst obo je inject .    2. tf Vee Z, deex, st Ge f)\&)= 2  Suppse 4 is not “sgecthe. sf B2e\%, Wei st GQ) \#z a    J mst be Sunective ;  f does nol hove t be Seetive . 

\vspace{10pt}

\par

\vspace{10pt}

    Ex 3.3-6 (1\% «tfeapt)    Let ZX >Y bea bryecte fametion , and let fo: YX be it inverse.  Verity the cancel [ton laws    SCH) =x fre all xe X ond LEG) = Fe all ye.    Grelude thet ft" ie abo invertible , and has f as its inverse ( thus (f7') ls +)    [. Fis bijecthe , st. We Y, dee X, 5. y= 76) (Svijecthe)  Vex eX, 1 \#m\% \& AX) + 7K) ( Lijectre)  Rem <7 Hi) = Axe)  > We y, Fl xeX, st. S\&) = y. Cat len one, svcectivity, at mest one injeeteity),  ad x= Sy). (Remark 3.321).  Ts (fof ')ly)= ¥.    = Vx,,4 € X, lk) ) Sl.) € Y,  from above. (Sef) \&:) =x,  (fFsF) G) = \%    2. Show Tat fois bijective.    since. f ic bietie, X\#X \& Hx.) \# Hs) 7 Wem EX,  (Jef )O) =X pe  (F'of) (.) = xX Thes f is injective .    Smee fis a Smation . Vee \%, al ye Y,    7, Sf YF SE).  Since Dis e Sanction, Vye ¥ axe x st. LC) =¥    Since f 1s bivective , Vye ¥ =! xe X, st Y= SC\,    Sis SF imese > Vee X Alye \% 54 SG)    f “ is bijective . M0 

\vspace{10pt}

\par

\vspace{10pt}

\begin{figure}[h]

\includegraphics[width=\textwidth]{/Users/nokman/git/pdftolatex/src/../localstore/trymeassets/9.jpg}

\centering

\end{figure}

\par

\vspace{10pt}

    Fx 3.3.¥ (@) Ce) Ci atterpt)  rf X és a subsel of Y hE best i XDY be He inchision map Fam XAT    deTred by mapping XH? X for all xéeX, ne, Lx sy 0) := xX ge all xe x.  The. map Ixox ‘Ss in porticelae called The jdeitity map on X,    (+) Shoy tt fF XESS DZ. Tim Tys2 e nor = U«n22.    (c) Show thet fLA > Bis bijecthe \_ Then fof 2 larg od  fof = lacy.    | TE XE YS 2B, thn df xeXx, xe¥, xe 2,  bevy \&) i= x, KEX, KET.  Lyo2 (X): =x, xe Y €€2Z, whee x = Teor).  The. Tr32° Vxar (x) := xX, xKEX, *€ a  = Zxo2).    2. If FAB is bijective. Then we brow thd foB> A.  fvof: fA 35A <= LArA  fef':B> B. Ve 8B. 

\vspace{10pt}

\par

\vspace{10pt}

\begin{figure}[h]

\includegraphics[width=\textwidth]{/Users/nokman/git/pdftolatex/src/../localstore/trymeassets/10.jpg}

\centering

\end{figure}

\par

\vspace{10pt}

    Fx 3.3.5 Ce) (latent)   Tf X is a svbsel of TV, bt bese i XY be He inchision map Fam XBT  deTned by mapping XX foe all xEX, fe, Levy XK) i= xX Pe allxe X,   The. map Ix-+x ‘sin porticela- called The jdeitity map on x,    (e) Show thet The Aypsthests that Xed Tare disemt f\& L “tly  can be dropped in td) if one odds He addifona( hypothesis change the  Conclsien  fot SE) “fo jor all xe€é xn Y. and the Appethesis  ‘s Inferchary e able.    /. Suppose Fg, Kee XO, we vat bh show LA: XVY > 2st. he Zxoxvr = 7  Pi XVZ Ge Xe. he Dry xey = 9.    2. Suppose A: XUY > ZS iis nek unrque , ot. Dh Xvr> 2. heh.    Zx> xox x? xuY hoa Lx 2 Kur : x— Z , st. AG) =f), vx.  h: xu¥Y >a 2  hii XU? \& he Dror ¥3B., Aly) = gly. Wyte    Craxer : Y> xUuYyY    Tf he Pome XD B, st h\&)= SG), YueX  hie Zyaxuy 2 YO B, sth = 7G) Wye ¥.  “he Alx) = LE) Vr e X Al«) £ hb. \&).    oa    Aly) fal I) Wyre a.    Thee is oly ono migne PuncBeon L This sgpher Te XY, «6 Ag. Woy 

\vspace{10pt}

\par

\vspace{10pt}

\begin{figure}[h]

\includegraphics[width=\textwidth]{/Users/nokman/git/pdftolatex/src/../localstore/trymeassets/11.jpg}

\centering

\end{figure}

\par

\vspace{10pt}

\begin{figure}[h]

\includegraphics[width=\textwidth]{/Users/nokman/git/pdftolatex/src/../localstore/trymeassets/12.jpg}

\centering

\end{figure}

\par

\vspace{10pt}

    Ex 3.4. ( (I attengt)   Let f: X>Tt be bijective farction lel Vk any subse of y.   Let F': YX be its inverse. fave Hot The frrnedl image a Vref!  is the same set as Te imerse image of V under .    LL Giwen fF: Yo X, Vey.  The forward image fv) := Ef "(y): yeV3 2X (Den 3.4.1)    2. Given F:X DY, Fis bisectivc frction, Leth be its verse, ssf.  Alv):= \$xeX: flelevh SX (Deh 3.45)    3. By definition of finctions, V2 ig ~~(V) = x, 21! ye Y\% st. Al2) = Y.  Omee fis the invege of f. S12) = Gof) fy) =y, where yeV.    + By def nition of Five fions, Vee ACV), al ye Y st AG) = y.  and. from Step 2, we know that ye VEY, sine his \#3 inverse.    5 From steps dard 4 we con see tet Vee fv), fa)ev,  Vege hlv), FG) ev.    Since Lis a biectwe fuctin. Wye V, Flee f),st fz) 7  Vy eV, !ge Alv), 57. 7G) zy    Vx, € fv), Vr « ACV),  A SCX.) = tn). Then X, =, Thvs xe £'), Xx, € AO).    X2EACV) , xref WV).  Tos Fv) shiv), Av) se flv). £7) =A, Sieh. 

\vspace{10pt}

\par

\vspace{10pt}

    Ex 3.4.2 (1etfemt’)   let f:X Y be a function. Let S be « sebsetof X. Leé U be astet of ¥  (5) hat com one say abst PCL) od S ingarem(?   Ci) whet oboat A(LCU)) md U2   Gis) whet aboxt F'FG AU) and Cu)?    IL (P74) (s) 2= 8xE Xs AK) E LG)\}, where Hs) 12 \$F): xe SB  Whefhee or not (F".f)Cs) 2 S depends en f.  Tf F ic injective, Ten (f'ef)(S\ \& S. (frm inbvitionr, not proved).  Tf F i sujecthe. Thee (fof) (5) £ S\$. (fram inbrition, nat proved).  HF is diet. Then (fF 'of)CS) = § (Ex 3.4.1 proved)    2. Pef")(v)i= \$f): xe fW)3 here \$“(u) = 3 xeX: 4G) ¢ U3.  Whether or mt (fof ')(V) © U depends on f  Ht fis injective . Then rot qrarmtee .  uti swjective Then (fof ")lV) 2U \& ast quaronTce (at proves J.  Tf Ff is biectee.  (fef"')(u) = VU (Ex 3.4. | proved)    3B (fle fo f')(u) = BxeX: Ase (fof)(U)§  Whe ther or not. (f'efef')lv) S f''W) depends ont.  xf f s inecBve Then net qeoconee .  Lf Fis sujectie not quarartee    zt fF is diectrr . (fitofof“')(V) = f~'lv) (form inh iter net proved] 

\vspace{10pt}

\par

\vspace{10pt}

\begin{figure}[h]

\includegraphics[width=\textwidth]{/Users/nokman/git/pdftolatex/src/../localstore/trymeassets/13.jpg}

\centering

\end{figure}

\par

\vspace{10pt}

\begin{figure}[h]

\includegraphics[width=\textwidth]{/Users/nokman/git/pdftolatex/src/../localstore/trymeassets/14.jpg}

\centering

\end{figure}

\par

\vspace{10pt}

\begin{figure}[h]

\includegraphics[width=\textwidth]{/Users/nokman/git/pdftolatex/src/../localstore/trymeassets/15.jpg}

\centering

\end{figure}

\par

\vspace{10pt}

    (1 "et fempt, checked others answer, Then abort)  Ex3.4. 6 (i) (2 attempt)   Ruwe Lemme 34 lo  Let X be a Sel, Then 3Y: T is c subset x3 1s a sel,    Tn othe. words, thero oxxts « sel 2,st. Ye F< YEX ‘ Be all objects ¥.    L Suppose we have a set FO, on sl. p. Axiem 3.1! (Prue set Axiom) .  we have f:X> 50,13. , fe £0,13”.    ?. From step | and Axiom 3.7 (Replacemert) , Suppose we have PE, ”) , 6.t fe each Se 30, 13\%  There ts at most one y jor which PG, y) is Trac.    Then tere ents « set \$y: Phy) is Tae fe some Fé \$0,13°3,  euch that fe any abject Zz:    ZE By: PY y) is True fr Some fe \$0,13\%3  <2 PY 2) is Tee rv some fe fo, 13”,    3. fom step 2. assona thet Py) te Tue tf y= f(813) £X, (w LOG?)  Then    zeSy: y= f (513), Fe \$2 13"5 <=) 2-f \&3)< x  bet 22\%, \$y: y= fC), fe 801373 = 2. Roved te Lemma. 

\vspace{10pt}

\par

\vspace{10pt}

    23.46 Ci) (atlapt)  Show that Axiom 2. [| can be dedwed , 1 using the preceding aXions of set Theor,  if one access Lemma 3.4.10 oS on axiom.    [We wart \% shou thet there extils a set ¥\% where LEV f2X7¥.(Aiom3.l)  Lriven it Lemma 3.4.10 is an axiom,  Such tek Ze ¥: ¥Ox3, if Ks asl.    2. From step! and Axion of renlacemerl , sppose ne hove PCY), ot,  for each Ye 2. there is af most on@ nc len Sar which PCYF) “ Tre  (if YES, Tm 1x74) is Tre \_ ofherv-lye. Sabe) ;    Then There cxisls x sek S42 AULA) is The foc some Ye 23 st  fer any Ainction ¥   LESH: POLL) is True fo some Ye \%S   =? UY, gy) is Tre far some YeS .    Then Since YEX, PULZ) is Tue for some FEZ  YeZ 7 TEX, ot. Wy 6 84: PTF), Ther\}    ote : ard nally > as Wye \&)+  CNet? Gobind "16 easier + pod, x £ vxe x, ve\%, 8 ve    7 From dep 2. me the lowe thee Vy 2 \$f PIA 2 Bf) AX\}.  Ths implies Axion 3.11, 

\vspace{10pt}

\par

\vspace{10pt}

    3    Ex 3.4.7 (Iathapt)   Let xX, ¥ be sets. Define a partial fonction tam X BY, f be.  any function Si X'—3 Y' whose domain X is a subset af X,   and whose co domain Yo is a sbset of Y.    Show that the callection of all partial finations srom X ts Y is itl « set.    by Axiom 3.11, we know thet there est a set y* 7. VE: X79 Y, fe y*  we alse know That tr every sbse€ xX’ of xX, and every subset Y of Y,  There exists a peetial AncTion h, st. Ar X'9 1.    8 step | and Axiom 3.7 (Replacemert) ; for any he ¥*, and any gubsels Xx’ and ¥'  suppose we have PCh, ¥\%' ) peelain’ns BA andl ¥'X |   geod thet fe each he Y\%, There is at most one YX st.   PLA, yx) is Ive, aka, hé yx"    They foe any et Q Qe STR: he, het hr A  > heQ , fhe Y\%.    Feem step 2. and Lemma 3.4:(0, re knw thet WRe SY: he Y*, he ¥*\$ =A  Qs”   and by Axiom 3.12.(Usion), we know fat all elmats in FY* she ¥*, he VS+A   ave sets.   Then there exist. a set UA whe ehmente ave elements »f ekmenk fA,    st. xe VA <=> (xe Q fe some QeA).  Ths Tle collection of qll partial factors tom Xt Y is itself a \_set. 

\vspace{10pt}

\par

\vspace{10pt}

    Ex 3.48 ([*atlempt)   Show tet Ar<ien 3-5 CRibinise Union) Con he deduced   Srom Axiom 3. ( (Sets ore ebjects), Arlem 3.4 (Singleton sets and pai sets),  avd Axiom 3.12 (Union) -   We wort te shew that xE AUB <=? (xeA or xe B). (Ariem 3.5 Peiewige Usion)    Let A be a sel, sf. all elements of A are sels.  By Axiom 3./2 (Union) , there exists a sel VA ,s.  xe UA => (xeS for Some SecA).    Sinilody , Phe iS anotle- sel B, with all the elements are sols,  By AMiem3.12, 6 UB <> (xES for some Se 8).    . From step |, thete are Tw sets VA and UB, apply Axiom 3,|, ve vould ask -   tt UAEUB, Tm xe VA > xe UB, xeUA)UWUB) 9 \& EA o xe B)  5 UVBEVA, -- - Both imply Axiom 3.5 (fairrise Union) .   Sf bith VUAEUB ond UB EVA:   “Then by Axiom 34, There exislh au set pB Pe SUA, UBS.    3. Fem step 3, we hwe a get P= £UA, VB3, ere VA od UB are sels,  Applying Aer 3.1 (Unin), thera exst o sot UP, 2.  xe VP c= (xe VA or xe UB) ce xe (VA)U (UB).  Tophing Axton 3.5 (Riise Union) 

\vspace{10pt}

\par

\vspace{10pt}

    E344 (|Tethenpt)  Show thet if B amd B’ are Tio elements of « set Z,  and To euch A € 1, we ass\{y> a set Ax. Men    F xe Ag: xe Ag Fr all he TS = § xeAge: xe Ag foe al LETS.  Stew thet defusClon of (Aa SxKe Ag > xé Ag Sor oll \& ez’ dosen't depend on B  (a4    Explain why this is Tet yé OO Ax <=? (ye Aa oe all he Z).  ékEL    I. By Axiom oplocemel, Ac each LET, suppese PCot, y) . sf  foc every Ae LT, there \& of ms one x st Roy) \& Tee  Pld. y) is Toe if yé Ag.  Then Were orice 2 gat. Bye yeAe, kOTY, st Lh any 2    Ze Sys ye Au He sme he I\$ <=) 26 An he some he L    2. From step|, we kim thet 2 Fy: yeAe foc Some KEL 4  <> 26€AL K some KEL,    Suppose. Te 5p,B°3, st SxeAg: xeAg foc all he £3  3 Ape 8x: xe Ache allde TS    \$xe Ag: xe Aw fe all Ke ZB  =? Ap’ = 3x: xe Aa fr all he Z3  “Thes Ag = Ag.  3, From skp 2, we con see thE (1) Ay:    AGT    gxeAg: xeAn fr all he L\$  3x : xeAg toc alke Ls does nof depend  on P.    u    4 And ly Steps Zand 3, ne can immvediade by see thet  yé OAs c=? (ye Ag he all LeL)    where the reglacement OXinr has heen applied . 

\vspace{10pt}

\par

\vspace{10pt}

\begin{figure}[h]

\includegraphics[width=\textwidth]{/Users/nokman/git/pdftolatex/src/../localstore/trymeassets/16.jpg}

\centering

\end{figure}

\par

\vspace{10pt}

    Citottemt, tired , brain not Anctioning)  Ex 3.411 Co" at fempt)  Let X bea sel, kt Lbeaw non - enply se€, and Ser all Xe tT,  let Ax bea sbset of X.              Slow tet X\U Ag = O (X\Ax) fend X\OAc= U (KV Aa)  ACI ACL aet det              Compare this mith De Megan \$ Laws Chapsition 3.1.27 0))  |, Recall th cbhiition of UAg iz UiAgi tel\}  MEL  X\ VAs :2 3xeX: xX¢ VAs 3 = jxeX i xd UfAg i re T\$  er ET    2. RB feplacement. axiom , yor every x ¢ U s Ax : ke ry and. omy svbset Q EX  Suppose we have Plx, 8). 5.7, for each xd U \$Aq:KeL\} Tore i, at myt  one Q fm whlch Pa) ‘3 Trae, (Pl«A) i Te of xEQS x).    Then ‘there exists a set JQ: x €Q Ar some x¢ UgAa de 3). such Cut  Loo any set 2,    Ze JQ: xe OA for some x USAg heT3\$  <=) xe BZ Ae some xf USAg: d€Z\}    3, Fromstep2, gence SQ: xe Q the some xg USha: herd os P, ot  we knew that Proteins all the sbseks ubich contains ot least one x f Ax, for Some eT    4. From steps 2 and 3, we Can cee thet UP = X\VAd  ez  5 Gosider ye OV (X\Au) <=? Lye X\Ad fr all Ke TZ)  del    And thom steps Bond 4, we can dedee that UP = \(X\ Ax). The A VA)    = X\WAK.  olen. 

\vspace{10pt}

\par

\vspace{10pt}

    Citattemyt, tired, brain not Aactimirg)  Ex 3.4. O”™ at fempt)  Let X hea set, ht Tbeaz non - emply set, and Sor all X¢ Z,  let An be a sehset of X.         Slow tet, X\U Ag = O (X\Ax) and [XVO Ag = U (KN Aa)  ACI ACL aet det              Grnpere This with De Nor gen § Lows Ce esi tiem 3.1.27)  X\OAg «= SKEX! KE NALS, where NAg = EX: xEAa fe allde LT\}  lé2 Kez lez  Since Vx € X\ NA« - feeXjamd PAu, for every deEL)  - By feplacement. axiom, yor every x fix xeAg fe alld an any abet Qe X  Suppose we hove Plx,\&), s.7, foe Caeh xd EK xCAG fe alldel Lore a awl myt  one Q foe ehtch Pl.B) x rae, (Pl) is tre sf xeQ€X).  Then ‘there exists a set 3Q: xER fc some X f SK: xeAg fralldeB? , such Gol  joc any set Z,  Ze 5Q : £€ A for some > AS Ki xeAy feallde Bd    <=) xe BZ Ae some x AS xixe Ay beled    ’ From step 2, gerele EQ 7 xe Q ofr some x \$x: xeAgto-allde TY as P, st.  we knew That P a alecins all the mbseke ubich contains orf lest one x ¢ Ad, fr all KE ZL.    From steps Zand 3, we can see thet UP = X\NAg  eZ  > Consider yé U (x\Aa) <=? (ye X\An Ar some KL)  A€T    And tom steps Bons! 4, ve. can dedew that UP = U(X\Ax) , Thes  x = UOUK\A  \f\ Aa ve \Ax) 

\vspace{10pt}

\par

\vspace{10pt}

    Ex 3.5.1G) (attempt)  Suppose. we define. the ordered pair Gey) fr any object x and y  by the formu (a, (x,y) is § §x\% , axyd\} (required servers | eapplioctims BI Axiom 3.4).  for exemple; (1,2) 1s the set \$513, \$1,233  Q,1) is the set £ 52\% 2,13  Cit) is the seé F534    Show thet such cx definition ( Kerclouske definition a an ordered pair) (defn 3.5: ‘)  obeys the properties of an ordered pote G5). (xy) = (xy) <=> (x=x' and y sy’).  |. (=) Soper (xy) = (ery).  Then by Koreloski \$ defn: ‘Team a an ore red. pair,  (ez) = \$543, fx, y33.  C4 ¥) = 33x, \$733,  sech thal \$43, Ex.y3 should fe an elemeat f \$403, 1x5 y/23, and  8\} £4Ly3 shoal be an element f £423, 4x,73\$    Compatng Siuglklon sel, and. pric sels respectively, 5X3 = gxn°3 \# \$x) 7'3.  exy3 = \$x, y's \# 3x3  Thess x 2x’, yey’    (<) Suppese x=x" and yey:  The by Kerahusks i defhifen on ortlered porns (x. y) = £ \$x3 \$x 2733,  = \$ 3x3, be 7 33. Aro~3. 4)  Ths. Car) (ery). 

\vspace{10pt}

\par

\vspace{10pt}

    Ex 3.5.16) (Dattengt)  Suppose. we define. the ordered pale Gy) Far any object x and y  by the formula (x,y) t= § fh, axyd\} (required servers | eapplioctoms a] Axiom 3.4),  for exemple, (1,2) 1s the set \$913, 51,233  Qt) is Be set § 52h, 82,133  Cit) is te seé F534    Show thet suck a definition ( KercTouske detwmition 4 an ordered pair) Cdefn 3S. ‘)  obeys the properties of an ordered pair G.5), (xy) = (x, y’) c= (=x! and y -y).    l (>)  A (xy) = (x) 7’) Ten S83, Spy th= LE, Ixy  “This implies thet 2x2 = Sx VEX ys = sx ZF sx, y'\$ as x '\#y’  Sxyh = 3x3 VEX. DD Fx ySH Ex’ os x\#Y  Ths x =x’ and yey!    2. (<=)  TE K=x' and 777" The Sxh = \$x'3 ond Sx, yh= Sx,y3.  Ths in (wy) : \$x\} and Sxy3 et, y) > ¢€ ily’) (y) £ (x‘y')  im Ky), §x3ed EtyRe ly) o> ey) Kye \& y)  Ths ly) = (2, y’) 

\vspace{10pt}

\par

\vspace{10pt}

\begin{figure}[h]

\includegraphics[width=\textwidth]{/Users/nokman/git/pdftolatex/src/../localstore/trymeassets/17.jpg}

\centering

\end{figure}

\par

\vspace{10pt}

\begin{figure}[h]

\includegraphics[width=\textwidth]{/Users/nokman/git/pdftolatex/src/../localstore/trymeassets/18.jpg}

\centering

\end{figure}

\par

\vspace{10pt}

    Fx 3.5.1 Ci) ([Matfopt)  Show that regacd less of fhe det nition of ordered pare,  the Cacte sian podet xX * Y of any Too sel xX, Y is ayein a set,    [. By oxiem f replace men, or every xe xX, onal any object \&,y)    Suppose PA, Gay) 1 Or pryety pertuirny x end Gy) st.  per euch x\& x, there he aC ma€ one ty) st. Pl bey) is Tre.    Nite: Axi) is Tae if ye ¥    Tan exirtia set. Llry) : Pll) is tue fowme xe XS,    tr any object 2,  ZeE Sty): ye ¥ for ome xé X3=A  —> PG,2) is Free fir ema xek    2. Fiom ep l, ve kmnw tet 2Ee A, E ae sls.  fir B= 3x, \$,y33, CShut detivition of ordered sefs),  Ths ly Aram cf nisa , thre ent, a sf UA 5.  ze UA > (ze (x.y) fr some (uve A)  <=> XxY 

\vspace{10pt}

\par

\vspace{10pt}

\begin{figure}[h]

\includegraphics[width=\textwidth]{/Users/nokman/git/pdftolatex/src/../localstore/trymeassets/19.jpg}

\centering

\end{figure}

\par

\vspace{10pt}

    aN Ss ij  x2\%, Xom, ts >    Ex 3.5.3 (|"etlenpt) mee,    Show thet, fhe detinitions of equalily for ordered pale and ordered. a- Tpke  are consig tent with the ceflerivity, symmeley, and Transitivity axioms .    (z the sense thet > Lf these axioms ore assumed fi hold for He individual components xy  of an ordered pac (x.y). Then they hold for the ordered paic helt ).    (x.y) = (uy) Hf (42\%, yey) Retest  (Xz) eign = CX:) ies en ff (x, HX, Xr eX), X= Xn) .    l+,y) > (x' 7’) At (x=x’ » 2 y') Symmetry = (<‘y7') = (x,y) .  (XN ie = (Ki) isin H (K=K, Kis Xf ye Xe 2x).  (xy) = (x, v'), (x4, y’) = (x"y”) Teensitialy,   (Xion = OX2) sien, CX iseen = (XD tecen 

\vspace{10pt}

\par

\vspace{10pt}

    Ex 3.5.4 (1" atlenpt )   Let A,B,C be sets   Show Det. Ax (BUC) = (AxB) U (Axc)  Show that, Ax (Bac) = (AxB) a (Axc)  Show thet Ax (B\C)-= LAxB)\(Ax6)    Ax (BUC) = Sty): xeA, ye Buck  (AxB) uv (Axc) = S(x,y): xe A,yeBsu f(xy): «eA, yer y simce A= A  af (xr) € Ax (BYO), Then xcA ye BUC  ya (x.y) é (AxB) v(Axc). Then xe AVA, ye BUC.    2. AxBnc)= Slxy): eA, ye Bnch  (A *B) 9 (AxC) = Fy): xe A, ye BS a Slay): xeAyech,  A ky) € Ax (BNC). Then (xe A, ye BNC)  Ff lay) € (AxB) a (Axc). Ten [xeA, yeBNC ) Since AA .    3. Ax (B\c) = SGy) 1 XE A, yeB\ch  AxB) \CAxc) = Sy): Ke A, ye BS\ Sry) | «eA, yeCh. Since A:A  FF lay) © Ax (BVO). Then (xe A, yeB\C)  F Gy)e (AB) \(Axd . Then (xe A, ved) ad (fA y¥C), 

\vspace{10pt}

\par

\vspace{10pt}

    E355 [|"ctfengt)   Let A,B,C, D be sets   Show Det (AxB)n (CxD) = CAac)x (Bnd)   Ts it te tet (AxB) u (cxd) = (Au) x(BuDd)?  Is it True tet (AxB)\ (xd) = CA\C) x (BD)?    |, (AxB)a (cxD) = Slay): xeA,ye BS nSGey)i x2 ¢ yD.  = Jay 2 x€ ANC, yeBnD\}    (Anc) « (BAD) = Sixy); xe ANC, yeBADS,  2. (AxB) U (cx D) = day): xe A, ye BS VU \$y) :xEC, yeds.    (AvC)« (BUD) = SG y): xe AUC, yeBUDS  = (AxB) UCCxD) U(A¥D) v (Cx B)    3. (AxB)\ (Cx D) = 3 Gay) sxe A, yeB3\ Sy): xeC, yeds.  = (xe A\C, ye B) v (KE ANG, ye B\D) v Cxe A, ye 810)  v (xeA, yeB) x (f ceA, ow DEB)  (A\C) « (B\D) = SGy): Xe A\C, ye B\D3 

\vspace{10pt}

\par

\vspace{10pt}

    Ex 3.5.6 (etfengt )    Let A,B,C, D be non- empty sets .   Show tht AxBe cxD iff ACC and BED   Show thet. AxB= CxD if A=C and B=D   What happens if some or all of the AypsTieses that the A,B,C, D ave nan empty , removed \%    /. (=)  AxBéeEcxD 3 (pg) e Flay): xe A, ye Bh . (pg)¢ ZG, y) + xeC, yeD3,  This implies thst Ak Cad BED.  Fifer A Zg C or B g D. Then conlradl Linn - AG.9) « AB, é cxD.    f) AsCaed BCD,  Tn xe A, xe GC, , ye B ye D  ze3lxy): x GA, ye35 => Be flay: xEC, yeD3,  AxC = B\&D,    2. \&)  AxB= cD ff AxBE CxD oa CxD SAG,  Ts ASC and BeDad CeAaw DEB,  A: C. BD,  (<=)  A=C. BeD \& Rec, ced, BEN, DEB.  AXBE CXD cud ZAK.  AXBs C\&D.    >. Suppose A= PD Ten AxBS CxD ff ASC, md BED remin Tre,  Bot mt AxB= GO HAC, BD,  os BCD, wad CoD sPW. 

\vspace{10pt}

\par

\vspace{10pt}

\begin{figure}[h]

\includegraphics[width=\textwidth]{/Users/nokman/git/pdftolatex/src/../localstore/trymeassets/20.jpg}

\centering

\end{figure}

\par

\vspace{10pt}

    Fx 3.5.§ (latempt )  Let Xi, Xn be sets, 5  Show thet the Grlestan product They Xe is emply iff at least one tf the Xi is enply,    |, ()    TTX: = J XK) igcen »x €X fr all leign\$  OO te XK eX eo « Xn.  Tidectin on n. Let n=2, Ff IX: « X*X\% ZB  Suppose Xr \#D, s-F. “   X«X\% = Jum): EX, mw eX. \} = D.  => AXE X, 9 K+ DB (Base Gre).    2. Fads tively assume thet\_ if IX bal x Then Ai, Ise fn. st Xx = oa  (WLoG). Let assvee that PRK, st es n, for The abe f held.  nt]  Sheu tee if TIX = GB, Tha AK = GW,    mt ” ,   JX. = TIX x Xo = DB, Wwe kmv that if TX =f. Then it holds  Supp ese Mie F D, st, TTX; Xr = 5 (XD) ¢ 5 enay : XL E X \{- all Ise Smt\}  . Then Xn = DP. Offense it dees ast Ill,   TIX, =P > WK jecer, Xi 2S  4  3 \&)  Suppose X = D, Then Similedy by inde tin TTX = (4 .  (She tet Xi 2OG = D  @ Assvme thet. Xr : p Ten Tx. c Z . Oflers Hor ~enty  ® Sho that Xati = g Te Tx. :D othes aan ot. 

\vspace{10pt}

\par

\vspace{10pt}

    Ex 3.5.4 (Mattengt)    Suppsse thet Land Tare tho sets, and fr all Le LT, kt Agden set,  for all BE LT kt Bp be a sel.   Ow Ja = U CA a8 .  Show That (VY Aa) a (Be) aarti 5p)    What happens if one infirchanget all the union and. itfersection symbols here. ®  [. Recall xe VAg <=7 (xe Ante some KET), UA = USA: de Th,  7+ (YA) 0 (U8) + UArterh 0 UPBp per  = \$xe UIA del\} : xe USBe peT33.  Q= U (AcaBp) = USAnBp + Gp)e Tes\}    Cpe Ixy : U SAa a Bp : (ple f@.p): der, pes\} \$    2. From step |, of xe P Then xe An NB for any AE I, any Bed.  x¥E Q. => PEQ  Hhexe€Q. Thr xe Aun Be ze ony (A,p) € Ix 5,  xéEP. 2 Acf    Ths P28    2. (OA )¥ (1 Bs) > \$x: xe A fer allke I U 2x: x é Bp f- ell pe TS.    1\ (Adu Bp) = \$x: xe AL UBs \& all (KPETxT\}  G.p)éeIky    Tey still open fF each other, oly f Lond J ove non enply 

\vspace{10pt}

\par

\vspace{10pt}

\begin{figure}[h]

\includegraphics[width=\textwidth]{/Users/nokman/git/pdftolatex/src/../localstore/trymeassets/21.jpg}

\centering

\end{figure}

\par

\vspace{10pt}

    Ex 3.5.10 Gi) Cl athengt )  Tf SiX>Y is a tnction, define fhe argh of f 4 be  The sbset of xx ¥ defined by § (x, 4@) ’ xeXh    Gi) Es avy obsel af XXY with the prpeely fet tor each xé X,  the ef Lye Ys lu) € GS has cxnetly ome elemert  Show thet Hoe is exastly one frelon AX Y whose gryh is egal f G.  ) kt Pt, y) be fhe Propels pitbin xandy, sT.  por each x EX, There is of ms€ one y , 5b.  ze Syek: lye Gk, where Ge Xx Y.    2. 4b skp | and axon \& replacement .  Hee exfia sf S2:P(x.2) a Tee te Some XE XS.  st te ony ore g.    ge A = sz: P.2) s Free tee seme xe Xk <> Plx-9) ce as I some xe xX,    3. Sup pete flee ave te functions fF and Ay pst. 4. 4x9 x, Arh, ne  § (x. AG) ixeX\$ = G  S\$ KAO): xex = G    Frm steps land 2, we know et. for very xeX tk) eA Ths fle) = hb)  AEA. -X.    Thes f: fy The “s onhy one fonction 

\vspace{10pt}

\par

\vspace{10pt}

    Ex3.5.10 (ii) Cl atferpt)  wie SiX> Y is a Sirction define The gregh aft b de  the shet of XY defied by § (x, 1\&)) sxe X\$.    (it) Svppsse we define x function £ t be am ortlered tiple f= (X,Y G) Vetial lie test  where X, Yara sels, and Gris a sebset sf Xx Y thet obeys the VeX, Zz! yer  vertiod le til. Le then detne the demain af such a Tiple f be X, filet  the co-domom fa be Y, «nd for every x \& X, we dethe Ax!) Like I.  He usioe ye Vist, lee G.   Show thet. this defnithes Gmpetille ttn Oh 33. i) and (Defe 3.3.6).    /, bie reed To chow that (Def. 2.3.[) Vee X Tye7 s-1, Lk)= y    Liefn3.3.8) Ay Af xeX, \&) = 96).  PK, X27.  2. Supyse Dy, yes y\# yr st Bl xe X, Ax): y, , A) = ya -K Collet  fre every KEX, fle)=y, y is vriqee te y) EG? f detrste ,  Ths, Y= 2.    jr CX, kk, Gr)    3. S»pprse TH 4 fs: (Xi, ¥,@.) it £29, Then X= Xe, 2h G+ @>  Holds . 

\vspace{10pt}

\par

\vspace{10pt}

    Fx 3.5.1 (Iethengt)  Show that Axiom 3.1( com mtuct he deduced fom Lemma 3.4(0  )    CRwerret  / Fee any Tho sels X ond Y, by defwitl 3S (CoAesinr prc), Common 3. \&=  Ke Viz ¥\& 7): xex, ye ¥\ Jb  by Lemma 24.10, Suppose G is = shel of Xx Ys. ~  There exists a sel QO, st. FO: GEXxFF=Q £3.50  Y  \&  2. Let P(),y) de fhe Propels pibin a andy Py an peat,    por cach ky) €Q There is o€ mes€ one y ST. co seh WV enls  We \$ ye Y: Plxy) is Teve For some Cx.) € QK | by axiom of replacemert .    3. From step 2, and the axiom f raplacemenl, oe every y in W,  gt, Suppose PC yf) be The prpety perTani y and Ff, st.  or each y € Wy, tee is af most one £, 11. PA zy, were Ply), x) 6 True (Step 2),  Then thre exist a sel. st, Soc any objeclh 2.    BELL: Ply A) is Tra te sme ye Wk    4 Fron skep 3, he Can see that sch a seb gheald contain all fomctlen  that. stisfies Defr 3.3. nd Defr 23.¥, and dedved fia presse Bron 

\vspace{10pt}

\par

\vspace{10pt}

    Fx 3.5.12 () (* o tlemgt)  Tis will estellish Frapositlon 2.].16 ( Recrsive Detniti-s) figeronsly , thet avoids cicevlacity.    (:) Let X Je a set. let fiN«X FX be a tnction.  Let c bean ekmet of xX.  Show Get Here ens a faction a:X 2X st. ale)= © and  alntt) = fn, aln\) Ar all ne IN, , wd frllemse bis fncth~ 4 Urge.    / Assume het This fonction a is an: XX, whee X= 3neiIN:in<NS  Fix N, and induc€ on n.  let n=O, Then ay (0) =< , an C1) = \#(0, we) ~ 40, <).    2. 5 ppese indvetivehy, ay fort) =A, ava) holds .  Stow thet Qn (@+) +t) = A (0+l, An (+1)) is abe valid.    Since Quln+s ) z an (otieX, Then Fae, an Gt) €exX.  Thes Qn (nt2) EX.    3. Suppose an: KX? X , Any -X> x, whardln \# Aurl,   Super An (NH) = ACN, an, (V)) valbids.   ard we know thet Gna, (N++) = a. Nee) |  GC then an (N+ 2) does ns€ exis  Ane (N42) exist, AL N\#1, Are (N\#1))  = HUN+l, FUN, ave LW)  > AN, HIN, av lN))  Ths it is migue as feo each Ne NV, N is wnique . 

\vspace{10pt}

\par

\vspace{10pt}

    Ex 3.5.12 G:) (etlenpt , check other 5 selvtion)  This vill establish Iropositlon 2.).16 (Rearsive Dedinitie-s) figgronsly    (ii) Frove G) without using any properties F the rcleral numbers  other than the Peano Axioms directly,    We wert FS show thet yor every ache nmbe- A/€ IN,  there exist a Unique pai- An, Bn At shsets of IN which obeys the Following prpeties :  La) An 1 BY =D (e) O€An Ce) Whenever né€ Bn ,we hve Art C Bn.  (b) Aw U Br =/N G) N+ € \&y (Ff) Whenever ne An and. az¢nNn, we have.  avec An  [. Trdvet on N. Left Nz O, Then An= \$0\% Bus \$1.2,3,.4...\$ =N\\$03  All canditfers ave satis fell.    2. Suppose indedively, that SIN, € IN, st NIN, soctisfas all condrtars.  Te tet Ni tt abe sectefies He couifions.  Bent = An U INttHS Brat = Bu \ SNe  An a Bun =P Ant U Bytt = IN, Of Avy \& OFAN  (N+t)+t © Buse. ete. All conditions aire. satistied.    3. From steps | and 2, gor every ack nmbe- MW € IN,   there exist a vniqve pai- An, Bn f subsets of IN which obeys The Plowing prpeties :  la) ANN BN? DB \&) oeAn Ce) Menever n€ Bu, we have Art 6 Bn.  (6) AnU Bn =/N G) Nr € ny (f) Whenever ne An and a¢tN, we have    atte An.    4 We com row sebsthile Aw as pne IN: néN3 nn “). 

\vspace{10pt}

\par

\vspace{10pt}

\begin{figure}[h]

\includegraphics[width=\textwidth]{/Users/nokman/git/pdftolatex/src/../localstore/trymeassets/22.jpg}

\centering

\end{figure}

\par

\vspace{10pt}

\begin{figure}[h]

\includegraphics[width=\textwidth]{/Users/nokman/git/pdftolatex/src/../localstore/trymeassets/23.jpg}

\centering

\end{figure}

\par

\vspace{10pt}

    Ex 3.5, 13 (|"atlengt )    Fipote of fhis “s 4 show, Got Thre is excell, ont verspn ot ne Lal number syslem in sel Theory,    Suppose we have a set IN of Hemalie aukel pumbers ; an "Herrnalhe Zen 0", and   an “atlernetve incremel opersLin * dich Tbec any alenalive roferal ambers n’e IN’   and relins ansther aHornetive nolera( nember nt’ \& IN “st The Rano Axiom (Axioms 2.| - 2.5),  all Asld vith the nealora number, Zero, and increment replaced Ju, theie affencLive. conlerpats ;    Shows thot There existe a bisection fi iN 2 IN from the nalora[ nembers vA    The. a/femative naliral mmbecs ot H(0) = 0 p and 3.0, Fr any né WN ard  ne IN“ , Wwe have Sb) « pn’ Af S (att) ante’    J. Suppese there exist « function fist fin IN’, £0) =o  he want b Show That it LO\} =p” Then Stars) =hte’ and the convene.  2. () af Ah)an’, tor ary ane IN ard any well’.    Tiducl on rn. Let hz O, Then ne kaw Tet it holds Hc) =O’    Seppe inducthely that SNe IN, st. FON) = NV’. Then A (N44) = V4"  Show thet He ) = Nit’ Thn LW +4) ++ ) = Weeder!  By Arion 2.2, if nelN, t. net EWN, oF.    F(N«): Nie" > Allvedes) = M, rce Mx (Nas) 44°  Traduction chsed .    2 (<=) of Het) = p'at Thr Als) = nr!  LC 2+ ©, Than st hel  Tdsctely atsme tat it i Tee Por n>/V.  TE \}yt)4t is Fe and if Wt) spot Fre SL Grobe.  (V7) is Fee.  Closed Ju fucten    Backim4s ad von, 

\vspace{10pt}

\par

\vspace{10pt}

    Ex 3.5.13 (2\% theme’)  Fipole of fhis “és 14 show Hol Thre is excell, ont versen ef ne bral number system in sel Theory,    Suppose we have o set. IN of lenatve nutkenl numbers ; an “aHersatte Zea" 0", and   an alernetve incremeL oper » hich tebee any olemalive nefera| anders n’e IN"   and relirns ansther aHprnetive nolera( nember t+’ \& IN “st. The Rano Axiom (Axioms 2.| - 2.5),  all Asld vith the nafora number, Zero, and increment replaced Ju, thetic alfereLive. conlerpats ;    Show thet. there exists a bisection fi iN IN from The nadir embers 1A  he. alfemtive naleral mombers ot flo) exe\} , and s-C. for any né IN ard  ne (N” we have flan! Af LOt+) = nae’.    injec  sgeeta\_    ports 

\vspace{10pt}

\par

\vspace{10pt}

\begin{figure}[h]

\includegraphics[width=\textwidth]{/Users/nokman/git/pdftolatex/src/../localstore/trymeassets/24.jpg}

\centering

\end{figure}

\par

\vspace{10pt}

\begin{figure}[h]

\includegraphics[width=\textwidth]{/Users/nokman/git/pdftolatex/src/../localstore/trymeassets/25.jpg}

\centering

\end{figure}

\par

\vspace{10pt}

    Ec3. 63 Yetfengt)  Let Aa bea nolan! number. let pf: fieN sls égn id IN be = Sanction .  Sew that here excts a nafiral nvmder M st. fli)Sm yor all KeSn.    L We know teat ANE le égn3 fir any nein soe Jule get (Debs 3.6.)>).    For (IN, we com see thet the st is ifoke AS supprte. if it is Au,  Then flee is always on adllitire/ elomot in IN, Hal does nt hove»  are- e- on Correspmderee in he forte ST, Sve ow bi jectfin do2e mat aout ,  So WW egal codrelty befnren Inthe sels and JN,    2. Seppe Gr sin, st Ahem ieises) = \& is.  ANEé CG, st. NeIN, waa /\/ hes fie lnayert val ,  Vye G, ye. Rt Sieg (NEN, ten TN" 6 Net = Ne  NsNn  Is Wye Gr, AN EIN. st. yen’ 

\vspace{10pt}

\par

\vspace{10pt}

\begin{figure}[h]

\includegraphics[width=\textwidth]{/Users/nokman/git/pdftolatex/src/../localstore/trymeassets/26.jpg}

\centering

\end{figure}

\par

\vspace{10pt}

\begin{figure}[h]

\includegraphics[width=\textwidth]{/Users/nokman/git/pdftolatex/src/../localstore/trymeassets/27.jpg}

\centering

\end{figure}

\par

\vspace{10pt}

\begin{figure}[h]

\includegraphics[width=\textwidth]{/Users/nokman/git/pdftolatex/src/../localstore/trymeassets/28.jpg}

\centering

\end{figure}

\par

\vspace{10pt}

\begin{figure}[h]

\includegraphics[width=\textwidth]{/Users/nokman/git/pdftolatex/src/../localstore/trymeassets/29.jpg}

\centering

\end{figure}

\par

\vspace{10pt}

    F3.6.8 C)Mottengt)   Let A and B be sets such Het there exis an injeclinn Z~A > B tom A 6B.  Assome A is nen: enphy, Show That here ouisls ox surjection gh > Atm Bo A,  bie wart thw tot Woe A, She Bot pa)>    /, Sinee. ve buen thet PAB . San inject, 6F.  Wa, a6 A, H\#RS fla) f Hla.) eB.  = A has n lester or ee cob ly tf. (Fe 36.7).  = \#(A)< \#8).    2. Fesm Slep [, Vie can SEE thet\_ tore. exbt o« Soalir- 4, 3.T.  eB, st. POarA.« a bigettn.  Ths U b ako a sugecton , 

\vspace{10pt}

\par

\vspace{10pt}

    Ex3.6.9 (Patlengt)  Let Aad B be Tite sels.  Show tint. AUB and ANB are alco frile sels,  and that \#(A) + \#(B) = ACAUB) + \# (ANB).    /. DE A md 1S we finite sels. Then LA > Siew: [<ign¥ fir some ne IN’.  sl. \#A)=n. This holds sill, £- B, st. kt 4B) =m.    2. TH AVB ond ANB we not fuse st, To Lee exist a bjyecton  betrem AUB amd IN -X- Corhediction. , smibly ith An@.  3. Finn, Drobec on m, Hf \# (A) =n. \#8) =m: O.  Then \#A) +4 (B) con  \#(B)=0 =7 B= GD AND=-DB  AvB= AVP = A.  Tes \# (Anf)= 2  \#lAvg)= a,    Ts Base Coe is Tre \_    4. Suppose atoticl, tot \#0) + \#(B)= HAVO + \# A086).  f \#@) =m.  Show that fF \#CB) = met - Thr te above still halds.  \# 6B) = mt imolies thet. B= By 3x3, Lemma 3.6).  Lf xed. “Thea AUB = AUB , ANB’ = AnB vu sxk  Ff x¢d A. Tn AUB'= AVBUSK3 , AN8’= ANB.  Ties. :fce A. \# (AUB): \#¢ (AUB), \#CANB) = \#(AnB) +|  Bef A. €CAB)= \#00) +] , \#ANBD = BCANPD.  Thes, CA) + \#03) = \# (AUB) + €(ANB) + | Lits -    \#CA)+ €(8) > \#A)t+ \#(0O*| RItS    Liducton closed - 

\vspace{10pt}

\par

\vspace{10pt}

    E23. Z./0 LI Hep    Let Ay, ..., An be file sets och fot \#(U Az) > 0. (figerbebe Principle)  Tho yn  Show tt there wists 3 € 3). \} such Hot \#OA:) 22  [. Recall U Ag: = 3Az 2 ce S004 Atm 312 Union).  éeY),..    Cerote. UA as RQ, smee. Avs An ore file ob.    pe kw that Qs alse a fife set. by Ex 3.6.4 and.    2 HA) - : \#(A) + \# (Ww), lee Wie OA p= 3x: xe A fr ie SI,.., 83  CES, m3,  2. From step |, Suppose thet Ale... An Gre all nen - empty sb,  vith one az: E A: : for each LE Jd] nf,  Ts FECA) + \#(m) = ZH) an  Suppose \#(~) = oO. st. we @. \#(0) = ZFA) =    3. From step 2    Tadnel on fn. Let n+ [, Q= Ar. \#@) = 240A) = \#4.) 2 |    Fer 2051 => HA)2 2 bere A= AUS*S = Sa,,x\%4  Suppae inductuely thet n= N, holds €, \#EC\> N =) \#(Q)~ VIF    =? Tee S[u4n3, stACZAU ERS,  Show that +his Jelds her n= Net, = Sa; xd.  \#(6) + N+t => \#(B) = Wei)ir > NO,  - 24h) OX = QU Ant  Tf \#Q) = WV. Then    Awe i+ Anas USX*3 => HCAwn) = [+t    H \#(@) =Nrt Tl from asi mation Some FZ! 7é \$]    08, 30 AH ALU BOL  Th iadoctgen closed. 

\vspace{10pt}

\par

\vspace{10pt}

\begin{figure}[h]

\includegraphics[width=\textwidth]{/Users/nokman/git/pdftolatex/src/../localstore/trymeassets/30.jpg}

\centering

\end{figure}

\par

\vspace{10pt}

\begin{figure}[h]

\includegraphics[width=\textwidth]{/Users/nokman/git/pdftolatex/src/../localstore/trymeassets/31.jpg}

\centering

\end{figure}

\par

\vspace{10pt}

    E3612) (tempt)  fir any neler ( mmbor n, hé Sr te the sel a all kjecTions    EX > X where X= SEEN: [e005    Gi) detwe. the pederial al fn paloral pum ber n reowsively by ol r= /  and (n+)! = (att) x nl fr all nakral num bers n    Show that \#\# (Sn) = al fe all nileru|, numbers rn.    As tronn ls), \# (Snr) = (att) x \#(S).  Te show \#(S) = al por all net, we reed show flere evist such Soe morphten .    |. det SWIM, ot Flo)anl, Lhe o frcbrel tntr -  Sanpete Plo) te th propty pert Phlis te f Fl = AGS).    2 Tet mn Let no, She  \#@Sc) =/.    3. Suppose inductively tht TNE ox. HN) = AG.) .    Show thet F(N+t) = \#(Suer ) .  we Ino Tht AiN+) = Oe)! = AN)~ ver).    rum previews “), ve fenow thet Hu.) = Wer) «x \#6.) .  st. SN++) = €CSr«:),    Ta tL cto, deed,    4. From ¢ 3, be lene thet Bere adsl a bije Sh. hefren Hr) crm BCG,  for all nafrul humdes p. 

\vspace{10pt}

\par

\vspace{10pt}

\end{document}
